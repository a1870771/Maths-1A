\documentclass[11pt]{article}
    \title{\textbf{Maths IA - Written 9}}
    \date{}
    \usepackage{amssymb}
    \usepackage{amsmath}
    \usepackage{multirow}
    \usepackage{graphicx}
	\graphicspath{ {./} }
    \author{Jack Larkin}
    \addtolength{\topmargin}{-3cm}
    \addtolength{\textheight}{3cm}
\begin{document}
\maketitle
\section*{Algebra}
\subsection*{(a)}
By adding non-negative slack variables to the given inequalities, the following linear system is obtained:
$$2x_1+x_2+x_3=4$$
$$2x_1+3x_2+x_4=6$$
\subsection*{(b)(i)}
The basic solution corresponding to the basic variables $x_3$ and $x_4$ is a somewhat trivial case. Upon converting the linear system to an augmented matrix, it becomes clear that pivots already exist in the corresponding collumns. $$\begin{bmatrix}
2 & 1 & 1 & 0 & 4\\
2 & 3 & 0 & 1 & 6\\
\end{bmatrix}$$
So setting all other variables equal to zero gives the solution $(0, 0, 4, 6)$.\\
\textbf{Answer:} $(0, 0, 4, 6)$
\subsection*{(ii)}
The basic solution corresponding to the basic variables $x_2$ and $x_4$ can be found with the same logic as the previous part, however with the added step of employing Gauss-Jordan elimination to convert the corresponding columns to pivot columns, if they aren't already, as follows. Beginning with the same augmented matrix:
$$\begin{bmatrix}
2 & 1 & 1 & 0 & 4\\
2 & 3 & 0 & 1 & 6\\
\end{bmatrix}$$
And applying the row operation $R_2 = R_2 - 3R_1$:
$$\begin{bmatrix}
2 & 1 & 1 & 0 & 4\\
-4 & 0 & -3 & 1 & -6\\
\end{bmatrix}$$
Gives the solution: $(0,4,0,-6)$.\\
\textbf{Answer:} $(0,4,0,-6)$
\subsection*{(iii)}
For the solution corresponding to $x_2$ and $x_3$, we can pick up where we left of with the matrix in the previous section.
$$\begin{bmatrix}
2 & 1 & 1 & 0 & 4\\
-4 & 0 & -3 & 1 & -6\\
\end{bmatrix}$$
$R_2=\frac{-1}{3}R_2$:
$$\begin{bmatrix}
2 & 1 & 1 & 0 & 4\\
\frac{4}{3} & 0 & 1 & \frac{-1}{3} & 2\\
\end{bmatrix}$$
$R_1=R_1-R_2:$
$$\begin{bmatrix}
\frac{2}{3} & 1 & 0 & \frac{1}{3} & 2\\
\frac{4}{3} & 0 & 1 & \frac{-1}{3} & 2\\
\end{bmatrix}$$
\textbf{Answer:} $(0,2,2,0)$
\subsection*{(c)}
The feasible region is defined by the additional constraints $x_1,x_2 \geq 0$, therefore the vertices of the region are given by the $x_1$ and $x_2$ coordinates of the basic solutions where each component is $\geq$ 0, i.e vertices exist at:
$$(0,0),(0,2),(2,0)\text{ and }(\frac{3}{2}, 1)$$
\section*{Calculus}
$$\int sin(\sqrt{x}) dx $$
Letting $u=\sqrt{x}$:
$$\therefore x = u^2$$
$$\therefore dx = 2udu$$
Which gives:
$$2 \times \int u\times sin(u) du $$
Applying integration by parts, letting $u=u$ and $sin(u)=v'$
$$\therefore u'=1 \text{ and } v=-cos(u)$$
$$\int u\times sin(u) du =\therefore -u\times cos(u)+\int cos(u)du$$
$$ = sin(u)-u\times cos(u)$$
$$ = sin(\sqrt{x})- \sqrt{x}cos(\sqrt{x})$$
Which gives the solution:
$$\int sin(\sqrt{x}) dx =2( sin(\sqrt{x})- \sqrt{x}cos(\sqrt{x}))+c$$
\end{document}