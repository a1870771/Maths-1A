\documentclass[11pt]{article}
    \title{\textbf{Maths IA - Written 5}}
    \date{}
    \usepackage{amssymb}
    \usepackage{amsmath}
    \author{Jack Larkin}
    \addtolength{\topmargin}{-3cm}
    \addtolength{\textheight}{3cm}
\begin{document}

\maketitle

\section*{Algebra}
To show that $V$ is a subspace of $\mathbf{R}^3$, three conditions need to be shown true:\\
\\1) The zero vector must be in $V$
\\2) If vectors $\mathbf{u,v}$ are in $V$, then $\mathbf{u+v} \in V$
\\3) If $\mathbf{u}\in V$, and $c \in \mathbf{R}$, then $c\mathbf{u}\in V$\\\\
Given the set $V=\{(x,y,z)\in\mathbf{R}^3|x-y-z=0\}$, it is clear that $(0,0,0)\in V$, as $(0-0-0=0)$, $\therefore$ the first condition is true.
Now if we take two solutions $\mathbf{u,v}$, then:
$$u_1-u_2-u_3=0 \;\;\text{and}\;\;v_1-v_2-v_3=0$$
Adding these two equations together, we get:
$$(u_1+v_1)-(u_2+v_2)-(u_3+v_3)=0$$
As required, and so the second condition is met.
Finally, if $c\in\mathbf{R}$, then:
$$cu_1-cu_2-cu_3=c(u_1-u_2-u_3)=0$$
So $c\mathbf{u}$ is a solution. This shows that V satisfies the three defining properties of a subspace.
\section*{Calculus}
Taking the derivative of the equation given for $\Delta A$ with respect to time $t$ yields the following:
$$\Delta A = \frac{1}{\pi}(\pi - 2arccos(\frac{a}{2r})+ \frac{a}{2r}\sqrt{4r^2-a^2}$$
$$\therefore \frac{d}{dt}\Delta A = \frac{1}{\pi} ( \frac{da}{dt}\frac{2}{\sqrt{1-\frac{a}{2r}^2}}+\frac{a}{2r^2}\frac{da}{dt}\frac{1}{2}(4r^2-a^2)^{\frac{-1}{2}}(-2a)+\frac{da}{dt}\frac{1}{2r^2}\sqrt{4r^2-a^2}) $$
Which is many things, short sharp and shiny not being ones that spring to mind. Simplifying this expression leads us to:
$$\therefore \frac{d}{dt}\Delta A = \frac{1}{\pi} ( \frac{da}{dt}\frac{2}{\sqrt{1-\frac{a}{2r}^2}}+\frac{da}{dt}(\frac{-2a^2}{4r^2\sqrt{4r^2-a^2}}+\frac{\sqrt{4r^2-a^2}}{2r^2} $$
$$ = \frac{1}{\pi} ( \frac{da}{dt}\frac{2}{\sqrt{1-\frac{a}{2r}^2}}+\frac{da}{dt}\frac{4r^2-2a^2}{2r^2\sqrt{4r^2-a^2}})$$
$$ = \frac{1}{\pi r}\frac{da}{dt}(\frac{1}{\sqrt{1-\frac{a}{2r}^2}}+\frac{2r^2-a^2}{r^2\sqrt{4r^2-a^2}})$$
\subsection*{(b)}
At the moment of totality, $a=r$ by definition, and it is given that the apparent radius $r=4$, which means the equation $\frac{d}{dt}\Delta A$ simplifies again significantly to: 
$$\frac{d}{dt}\Delta A = \frac{1}{\pi}\frac{da}{dt}(\frac{1}{\sqrt{1-\frac{4}{2\times 4}^2}}+\frac{2\times 4^2-4^2}{4^2\sqrt{4\times 4^2-4^2}})$$
$$ = \frac{1}{\pi}\frac{da}{dt}(\frac{3\sqrt{3}}{4})$$
Defining $t$ as hours since the moment when the edges of the sun and the moon first touch, consider how the distance between the centre of each circle is at its greatest at $t=0$ and $t=3$, and that at the midpoint of the eclipse, $t=1.5$, $a$ is 0.
This is enough to define $a(t)$, and consequently $a'(t)$ as follows:
$$a(t)=\frac{4r}{3}t-2r$$
$$a'(t)=\frac{4r}{3}$$
$$a'(t)=\frac{16}{3}$$
Bearing in mind that velocity is negligibly close to constant, for this case. 

Which is all the information needed to claim that
$$\Delta A  = \frac{4\sqrt{3}}{3\pi}$$

\end{document}