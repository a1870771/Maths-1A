\documentclass[11pt]{article}
    \title{\textbf{Maths IA - Written 4}}
    \date{}
    \usepackage{amssymb}
    \usepackage{amsmath}
    \author{Jack Larkin}
    \addtolength{\topmargin}{-3cm}
    \addtolength{\textheight}{3cm}
\begin{document}

\maketitle

\section*{Algebra}
\subsection*{(a)}
To determine the value(s) of $a$ for which the given vectors are linearly dependant, we need to show that the system of linear equations which they represent has a non-trivial solution, the system of equations being:
$$x1
\begin{bmatrix}
1\\
0\\
2\\
\end{bmatrix}
+ x2
\begin{bmatrix}
1\\
1\\
a\\
\end{bmatrix}
+ x3
\begin{bmatrix}
a\\
1\\
-1\\
\end{bmatrix}
=
\begin{bmatrix}
0\\
0\\
0\\
\end{bmatrix}
$$
Expanding this equation gives:
$$x_1 + x_2 + ax_3 = 0$$
$$x_2 + x_3 = 0$$
$$2x_1 + ax_2 -x_3 = 0$$
Which can be written as follows as an augmented matrix:
$$\begin{bmatrix}
1 & 1 & a & 0\\
0 & 1 & 1 & 0\\
2 & a & -1 & 0\\
\end{bmatrix}$$
Which, via Gauss-Jordan elimination, can be written in row echelon form as follows:

$$\begin{bmatrix}
1 & 1 & a & 0\\
0 & 1 & 1 & 0\\
0 & 0 & -3a+1 & 0\\
\end{bmatrix}$$
From which it can be observed that $x_3(-3a+1)=0$, and therefore either $x_3=0$ or $a=\frac{1}{3}$.\\\\
If we let $x_2=0$, we will arrive at a trivial solution ($x_1 = x_2 = x_3 = 0$), so the only value of $a$ for which the given vectors are linearly dependant is $\frac{1}{3}$.\\
\\\textbf{Answer: } $a = \frac{1}{3}$.
\subsection*{(b)}
The answer in the previous section gives us the ability to rewrite the vectors in row-reduced matrix as follows:
$$\begin{bmatrix}
1 & 0 & \frac{-2}{3} & 0\\
0 & 1 & 1 & 0\\
0 & 0 & 0 & 0\\
\end{bmatrix}$$
Which then lets us write $v_3$ as a linear combination of $v_1$ and $v_2$ by simple observation:
$$v_3=\frac{-2}{3}v_1+v_2$$

\section*{Calculus}
\subsection*{(a)}
Given $x^3=x^2-3y^2$, $\frac{dy}{dx}$ can be found with implicit differentiation.
The first step is to differentiate both sides of the equation:
$$3x^2=2x-6y\frac{dy}{dx}$$
and then simply isolate $\frac{dy}{dx}$ as the subject of the equation:
$$\frac{dy}{dx}=-\frac{3x^2-2x}{6y}$$
\subsection*{(b)}
At the point on the Tschirnhausen cubic where $x=-2$ and $y>0$, $y=2$. Therefore the slope of the tangent to the curv is given by $\frac{dy}{dx}$ at $(x,y)=(-2,2)$.
$$\frac{dy}{dx}=-\frac{3(-2^2)-2(-2)}{6(2)}$$
$$\therefore \frac{dy}{dx}=-\frac{4}{3}$$
\end{document}