\documentclass[11pt]{article}
    \title{\textbf{Maths IA - Written 11}}
    \date{}
    \usepackage{amssymb}
    \usepackage{amsmath}
    \usepackage{multirow}
    \usepackage{graphicx}
	\graphicspath{ {./} }
    \author{Jack Larkin}
    \addtolength{\topmargin}{-3cm}
    \addtolength{\textheight}{3cm}
\begin{document}
\maketitle
\section*{Algebra}
\subsection*{(a)}
To begin with this problem, we can consider $(A-2I)\begin{bmatrix}1\\1\\1\\\end{bmatrix}$. Utilising the distributive properties of matrix multiplication, this becomes $A\begin{bmatrix}1\\1\\1\\\end{bmatrix}-2I\begin{bmatrix}1\\1\\1\\\end{bmatrix}$. It is given that the first term is equal to $\begin{bmatrix}2\\2\\2\\\end{bmatrix}$, and the identity matrix is known, therefore the following statement should be observed:
$$(A-2I)\begin{bmatrix}1\\1\\1\\\end{bmatrix}=\begin{bmatrix}2\\2\\2\end{bmatrix}-2\begin{bmatrix}1\\1\\1\\\end{bmatrix}$$
$$=\begin{bmatrix}2\\2\\2\end{bmatrix}-\begin{bmatrix}2\\2\\2\end{bmatrix}$$
$$=\begin{bmatrix}0\\0\\0\end{bmatrix}$$
Therefore, the homogenous linear system $(A-2I)\mathbf{X}=0$ has a solution which isn't the trivial solution $\mathbf{X=0}$. By Theorem 13.3, since this is the case, it is also the case that $\mathbf{A-2I}$ is not invertible, and furthermore, by Theorem 18.3, the determinant of $\mathbf{A-2I}$ must be equal to 0.\\\\
\textbf{Answer:} 0
\subsection*{(b)}
To calculate the determinant of the given matrix $B$, the elementary row operations used to turn $A$ into $B$ must be defined, as this will let the matrix $B$ be expressed in terms of matrices with known determinants. 
$$A= \begin{bmatrix}
a & b & c \\
d & e & f \\
g & h & i \\
\end{bmatrix}$$
$R_3=2R_3$:
$$= \begin{bmatrix}
a & b & c \\
d & e & f \\
2g & 2h & 2i \\
\end{bmatrix}$$
$R_2=R_2+2R_1$
$$= \begin{bmatrix}
a & b & c \\
d + 2a & e + 2a & f + 2a \\
2g & 2h & 2i \\
\end{bmatrix}$$
\\$$=B$$\\
$$\therefore B = \begin{bmatrix}
1 & 0 & 0 \\
0 & 1 & 0 \\
0 & 0 & 2 \\
\end{bmatrix}
\begin{bmatrix}
1 & 0 & 0 \\
2 & 1 & 0 \\
0 & 0 & 1 \\
\end{bmatrix}
\begin{bmatrix}
a & b & c \\
d & e & f \\
g & h & i \\
\end{bmatrix}$$\\
And from Theorem 18.3 (b), it should be observed that the determinant of the products is equal to the product of the determinants when each matrix is the same size.
$$\therefore det(B)=
\begin{vmatrix}
1 & 0 & 0 \\
0 & 1 & 0 \\
0 & 0 & 2 \\
\end{vmatrix}
\begin{vmatrix}
1 & 0 & 0 \\
2 & 1 & 0 \\
0 & 0 & 1 \\
\end{vmatrix}
\begin{vmatrix}
a & b & c \\
d & e & f \\
g & h & i \\
\end{vmatrix}
$$
$$=2\times 1\times 3$$
$$=6$$
\textbf{Answer: }6
\section*{Calculus}
The first step in evaluating this integral is finding the limit, if it exists, as $x$ approaches infinity in the integrand:
$$\lim_{x\to \infty} \frac{1}{\sqrt{x}+x\sqrt{x}} = 0$$
Since the limit exists, using definition 3.3, the given integral can be redefined as:
$$\lim_{b\to \infty}\int_1^b \frac{1}{\sqrt{x}+x\sqrt{x}}dx$$
Using the given substition $u=\sqrt{x}$, can be written as:
$$\int_1^\infty \frac{2u}{u+u^3}du$$
Note that the upper and lower bounds do not change in this case. This simplifies to:
$$2\int_1^\infty \frac{1}{1+u^2}du$$
Which is of the form of a common integral and hence becomes:
$$2\int_1^\infty arctan(u)du$$
Which can be solved for an exact value as follows:
$$2\int_1^\infty arctan(u)du=2[arctan(\infty)-arctan(1)]$$
$$=2(\frac{\pi}{2}-\frac{\pi}{4})$$
$$=\frac{\pi}{2}$$
\end{document}