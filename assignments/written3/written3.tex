\documentclass[11pt]{article}
    \title{\textbf{Maths IA - Written 3}}
    \date{}
    \usepackage{amssymb}
    \usepackage{amsmath}
    \author{Jack Larkin}
    \addtolength{\topmargin}{-3cm}
    \addtolength{\textheight}{3cm}
\begin{document}

\maketitle

\section*{Algebra}
To determine the linear equation that characterizes whether $\mathbf{x}$ is in the span of the given vectors, we can set up an augmented matrix and row reduce it to determine whether the system is consistent or inconsistent.
\\
We start by writing the augmented matrix:
$$\begin{bmatrix}
1 & 0 & 1 & | & x_1 \\
-2 & 1 & 0 & | & x_2 \\
-1 & 1 & 1 & | & x_3 \\
\end{bmatrix}$$ \\
We then apply row operations to simplify the matrix. Adding 2 times row 1 to row 2 gives:
$$\begin{bmatrix}
1 & 0 & 1 & | & x_1 \\
0 & 1 & 2 & | & 2x_1 + x_2 \\
-1 & 1 & 1 & | & x_3 \\
\end{bmatrix}$$ \\
Adding row 1 to row 3 gives:
$$\begin{bmatrix}
1 & 0 & 1 & | & x_1 \\
0 & 1 & 2 & | & 2x_1 + x_2 \\
0 & 1 & 2 & | & x_1 + x_3 \\
\end{bmatrix}$$ \\
Subtracting row 2 from row 3 gives:
$$\begin{bmatrix}
1 & 0 & 1 & | & x_1 \\
0 & 1 & 2 & | & 2x_1 + x_2 \\
0 & 0 & 0 & | & x_1 + x_3 - 2x_1 -x_2\\
\end{bmatrix}$$ \\
Simplifying the last row gives:
$$\begin{bmatrix}
1 & 0 & 1 & | & x_1 \\
0 & 1 & 2 & | & 2x_1 + x_2 \\
0 & 0 & 0 & | & -x_1 -x_2 + x_3\\
\end{bmatrix}$$ \\
From this, we can see that the system is consistent if and only if $-x_1 - x_2 + x_3 = 0$. Therefore, the linear equation that characterizes whether x is in the span of the given vectors is $-x_1 - x_2 + x_3 = 0$. In other words, if the components of $\mathbf{x}$ satisfy this equation, then $\mathbf{x}$ is in the span of the given vectors, and if they do not satisfy this equation, then $\mathbf{x}$ is not in the span.


\section*{Calculus}
Calculating the left hand limit of $f(x)$ at $x=-1$ gives the following:
$$\lim_{x\to -1^-}f(x)=(-1)^2+2(-1)=-1$$
So for $f(x)$ to be continuous, it must be defined at the point $(-1,-1)$, and therefore $a(-1)+b=-1$.
\\\\
Now finding the right hand limit at $x=0$ gives the following:
$$\lim_{x\to 0^+}f(x)=sin(0)=0$$
Therefore, for $f(x)$ to be continuous, it must also be defined at hte point $(0,0)$. This means that $f(x)=ax+b$ forms a straight line which connects the points $(-1,-1)$ and $(0,0)$. Now it can be ascertained through simple observation that the y-intercept of this line is $0$, and the gradient is $1$, therefore $a=1$ and $b=0$.

\end{document}

